\documentclass{article}
\usepackage{enumitem}
\usepackage{hyperref}

\begin{document}

\section*{1. Project Overview}
\textbf{Project Description:} The Twitter clone project aims to create a social media platform where users can share short messages, follow other users, and engage with content through likes and comments. \\
\textbf{Database Purpose:} The database serves as the backend storage for user accounts, tweets, user interactions, and other related data. \\
\textbf{Stakeholders:} Developers, Product Managers, Designers, Users

\section*{2. Database Management System (DBMS)}
\textbf{DBMS:} PostgreSQL Version 16.2 (Supabase)

\section*{3. Database Schema}
\textbf{Schema Structure:} Follows a relational database model \\
\textbf{Tables and Views:}
\begin{itemize}[label=--]
    \item Users
    \item Tweets
    \item Likes
    \item Comments
    \item Followers
    \item Notifications
    \item Retweets
    \item Notification\_Types
    \item Saves
    \item Profile
    \item Tags
\end{itemize}

\section*{4. Entities and Attributes}
\textbf{Comments:} This table stores comments made by users on tweets.
\begin{itemize}[label=--]
    \item id: Unique identifier for each comment.
    \item tweet id: Foreign key referencing the tweet on which the comment was made.
    \item user id: Foreign key referencing the user who posted the comment.
    \item content: Text content of the comment.
    \item created at: Timestamp indicating when the comment was posted.
\end{itemize}
\textbf{Followers:} This table stores the relationships between users who follow each other.
\begin{itemize}[label=--]
    \item id: Unique identifier for each follower relationship.
    \item follower id: Foreign key referencing the user who is following.
    \item following id: Foreign key referencing the user who is being followed.
    \item created at: Timestamp indicating when the following relationship was established.
\end{itemize}
\textbf{Likes:} This table stores the likes given by users to tweets.
\begin{itemize}[label=--]
    \item id: Unique identifier for each like.
    \item tweet id: Foreign key referencing the tweet that was liked.
    \item user id: Foreign key referencing the user who liked the tweet.
    \item created at: Timestamp indicating when the like was given.
\end{itemize}
\textbf{Notification Types:} This table stores different types of notifications.
\begin{itemize}[label=--]
    \item id: Unique identifier for each notification type.
    \item name: Name or description of the notification type.
\end{itemize}
\textbf{Notifications:} This table stores notifications sent to users.
\begin{itemize}[label=--]
    \item id: Unique identifier for each notification.
    \item user id: Foreign key referencing the user who received the notification.
    \item type id: Foreign key referencing the type of notification.
    \item content: Text content of the notification.
    \item created at: Timestamp indicating when the notification was sent.
\end{itemize}
\textbf{Profile:} This table stores user profiles.
\begin{itemize}[label=--]
    \item id: Unique identifier for each user profile.
    \item user id: Foreign key referencing the user.
    \item bio: Biography or description of the user.
    \item location: Location of the user.
    \item website: Website URL of the user.
    \item created at: Timestamp indicating when the profile was created.
\end{itemize}
\textbf{Retweets:} This table stores retweets made by users.
\begin{itemize}[label=--]
    \item id: Unique identifier for each retweet.
    \item tweet id: Foreign key referencing the original tweet being retweeted.
    \item user id: Foreign key referencing the user who retweeted the tweet.
    \item created at: Timestamp indicating when the retweet was made.
\end{itemize}
\textbf{Saves:} This table stores saved tweets by users.
\begin{itemize}[label=--]
    \item id: Unique identifier for each saved tweet.
    \item tweet id: Foreign key referencing the tweet being saved.
    \item user id: Foreign key referencing the user who saved the tweet.
    \item created at: Timestamp indicating when the tweet was saved.
\end{itemize}
\textbf{Tweets:} This table stores individual tweets posted by users.
\begin{itemize}[label=--]
    \item id: Unique identifier for each tweet.
    \item user id: Foreign key referencing the user who posted the tweet.
    \item imgUrl: URL link that references the image that's attached to the tweet.
    \item likeCount: Number of likes the tweet received.
    \item retweetCount: Number of retweets the tweet received.
    \item tag: Other users that have been mentioned in the tweet.
    \item content: Text content of the tweet.
    \item created at: Timestamp indicating when the tweet was posted.
\end{itemize}
\textbf{User:} This table stores user accounts.
\begin{itemize}[label=--]
    \item id: Unique identifier for each user.
    \item username: Username chosen by the user (unique).
    \item email: Email address of the user (unique).
    \item password: Encrypted password of the user.
    \item created at: Timestamp indicating when the user account was created.
\end{itemize}

\section*{10. Backup and Recovery}
\textbf{Backup Strategy:} Regular backups scheduled via Supabase \\
\textbf{Recovery Procedures:} Restore from backups in case of data loss or corruption

\section*{11. Scaling and Performance}
\textbf{Scaling:} Horizontal scaling with Supabase's auto-scaling capabilities \\
\textbf{Performance Tuning:} Index optimization and query optimization for better performance

\section*{12. Data Migration and Integration}
\textbf{Data Migration:} Import existing user data if applicable \\
\textbf{Integrations:} Integration with third-party services such as Google and GitHub for authentication.

\section*{13. Data Retention and Archiving}
\textbf{Retention Policies:} Retain user data as long as the account exists \\
\textbf{Archiving:} Archive inactive accounts or delete upon user request

\section*{14. Documentation and Maintenance}
\textbf{Maintenance Procedures:} Regular database schema updates and maintenance \\
\textbf{Documentation:} Keep database schema documentation up-to-date

\section*{15. Dependencies}
\textbf{External Dependencies:} Supabase SDK for database interactions

\section*{16. Testing and Quality Assurance}
\textbf{Testing Procedures:}
\begin{itemize}[label=--]
    \item Unit tests for database functions and procedures
    \item Integration tests for database interactions
\end{itemize}

\section*{17. References}
\textbf{Supabase Documentation:} \url{https://supabase.com/docs} \\
\textbf{Supabase Database Documentation:} \url{https://supabase.com/docs/guides/database/overview}

\end{document}

