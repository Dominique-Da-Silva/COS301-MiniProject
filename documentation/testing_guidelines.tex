\documentclass{article}
\usepackage{enumitem}

\begin{document}

\section{End-to-End Testing with Cypress}

\subsection{Test Planning:}
\begin{enumerate}[label=\arabic*.]
    \item Define the scope of the E2E tests, including the features and functionalities to be tested.
    \item Identify the user flows and scenarios to be covered by the tests.
    \item Determine the environments (e.g., development, staging, production) where the tests will be executed.
\end{enumerate}

\subsection{Test Design:}
\begin{enumerate}[label=\arabic*.]
    \item Write test cases for each user flow, starting from the initial interaction to the expected outcome.
    \item Design test scenarios that cover both positive and negative scenarios, including edge cases and error conditions.
    \item Use Cypress commands to interact with the application UI elements, such as clicking buttons, entering text, and verifying elements.
\end{enumerate}

\subsection{Test Execution:}
\begin{enumerate}[label=\arabic*.]
    \item Set up Cypress environment and configure test runner.
    \item Run the E2E tests against the application, either locally or on a test server.
    \item Monitor test execution and analyze test results for any failures or errors.
    \item Debug failing tests by reviewing logs, screenshots, and videos captured during test execution.
    \item Refine test cases and re-run tests as needed to ensure reliability and accuracy.
\end{enumerate}

\subsection{Results Analysis:}
\begin{enumerate}[label=\arabic*.]
    \item Review test results to identify any failures, errors, or unexpected behavior.
    \item Analyze the root cause of failures and determine if they are due to issues in the application code or test scripts.
    \item Prioritize and triage issues based on severity and impact on the application.
    \item Communicate findings with the development team and stakeholders to facilitate resolution and mitigation of issues.
\end{enumerate}

\subsection{Types of End-to-End Testing:}
\begin{enumerate}[label=\arabic*.]
    \item \textbf{Horizontal E2E Test:} Focuses on validating the application from the end user's perspective, covering complete user journeys and interactions.
    \item \textbf{Vertical E2E Test:} Focuses on testing specific layers or components of the application stack, such as API endpoints, database interactions, and backend services.
\end{enumerate}

\subsection{Cypress (UI-testing and E2E testing):}
\begin{enumerate}[label=\arabic*.]
    \item Cypress is a powerful testing framework for conducting both UI testing and end-to-end testing of web applications.
    \item It provides a rich set of commands and assertions for interacting with web elements and validating application behavior.
    \item Cypress offers features like automatic waiting, real-time reloading, and built-in test runner for efficient test development and execution.
\end{enumerate}

\subsection{Unit Tests:}
\begin{enumerate}[label=\arabic*.]
    \item Unit testing involves testing individual units of code, such as functions, methods, and classes, in isolation from the rest of the application.
    \item It focuses on validating the correctness, robustness, and reliability of the code at a granular level.
    \item Unit tests help identify defects early in the development process and improve code quality and maintainability.
\end{enumerate}

\subsection{Logic Checks:}
\begin{enumerate}[label=\arabic*.]
    \item Logic checks involve verifying the correctness of logical operations and algorithms within individual units of code.
    \item Unit tests for logic checks assess whether the code produces the expected output for different input scenarios and edge cases.
\end{enumerate}

\subsection{Boundary Checks:}
\begin{enumerate}[label=\arabic*.]
    \item Boundary checks focus on testing the behavior of code at the boundaries of acceptable input values or conditions.
    \item Unit tests for boundary checks assess how the code handles edge cases and boundary conditions to ensure robustness and reliability.
\end{enumerate}

\subsection{Error Handling:}
\begin{enumerate}[label=\arabic*.]
    \item Error handling tests evaluate the effectiveness of error detection and recovery mechanisms within units of code.
    \item Unit tests for error handling assess whether the code correctly identifies and handles exceptions, maintains system integrity, and prevents data corruption.
\end{enumerate}

\subsection{Object-oriented Checks:}
\begin{enumerate}[label=\arabic*.]
    \item Object-oriented checks focus on verifying the behavior of classes, methods, and objects within an object-oriented software system.
    \item Unit tests for object-oriented checks assess whether classes and objects adhere to object-oriented design principles and maintain state and behavior as intended.
\end{enumerate}

\subsection{Vitest (Functional, Unit, and Integration Testing):}
\begin{enumerate}[label=\arabic*.]
    \item Vitest is a comprehensive testing framework that combines functional, unit, and integration testing methodologies.
    \item It provides a unified approach to testing software systems, covering a wide range of testing scenarios and use cases.
    \item Vitest helps ensure the quality, reliability, and performance of software applications across different layers and components.
\end{enumerate}
\subsection*{Jest (Unit Testing):}
\begin{enumerate}[label=\arabic*.]
    \item Jest is a popular JavaScript testing framework that specializes in unit testing of JavaScript code.
    \item It provides a simple and intuitive API for writing test cases, running tests, and generating test reports.
    \item Jest offers features like snapshot testing, mocking, and code coverage analysis to enhance the quality and effectiveness of unit tests.
    \item All unit testing will be done in Jest. 
    \item Jest is a zero-configuration testing platform that can be used to test any JavaScript code.
    \item Advantages of Jest are: provides CLI tool to control tests with ease, provides syntax to test a single test or to skip tests, provides a way to run tests in parallel, and provides a way to run tests in watch mode and it offers code coverage through CLI.
    \item One of the main features are Snapshots which are key for frontend testing as they allow the tester to verify the integrity of large objects. It is a way to test the output of a component and compare it to a saved snapshot.

\end{document}
